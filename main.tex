\documentclass{article}
\usepackage[utf8]{inputenc}
\usepackage{booktabs}
\usepackage{amsmath}
\usepackage{geometry}
\geometry{margin=1in}
\usepackage[spanish]{babel}
\usepackage{graphicx}
\geometry{margin=2.5cm}

\title{Informe}
\author{Samuel Primera C.I: 31.129.684, Samuel Reyna CI: 30.210.759}
\date{8 de Julio del 2025}

\begin{document}


\maketitle

\section{Cómo se implementa la recursividad en MIPS32}
Cada llamada recursiva en MIPS32 crea un nuevo marco de pila para aislar variables locales, argumentos y la dirección de retorno:

\begin{enumerate}
  \item \textbf{Al entrar en la función}:
    \begin{itemize}
      \item Reservar espacio en pila: \texttt{addi \$sp, \$sp, -X}.
      \item Guardar registros callee-saved (registros preservados por la función llamada) y \texttt{\$ra}:
        \begin{verbatim}
sw \$ra, 0(\$sp)
sw \$s0, 4(\$sp)
        \end{verbatim}
      \item Copiar argumentos o variables locales adicionales en la pila.
    \end{itemize}

  \item \textbf{Al hacer la llamada recursiva}:
    \begin{itemize}
      \item Cargar nuevos argumentos en \texttt{\$a0}--\texttt{\$a3}.
      \item Ejecutar \texttt{jal NombreFunción}, que guarda en \texttt{\$ra} la dirección de retorno.
    \end{itemize}

  \item \textbf{Al regresar de la llamada}:
    \begin{itemize}
      \item Restaurar registros desde la pila:
        \begin{verbatim}
lw \$ra, 0(\$sp)
lw \$s0, 4(\$sp)
        \end{verbatim}
      \item Liberar espacio en pila: \texttt{addi \$sp, \$sp, X}.
      \item Saltar a la dirección de retorno con \texttt{jr \$ra}.
    \end{itemize}
\end{enumerate}

El registro \texttt{\$sp} mantiene el puntero de pila, otorgando un área propia a cada nivel de recursión.



\section{Riesgos de desbordamiento y cómo mitigarlos}


\begin{itemize}
  \item La recursión profunda puede agotar la memoria de pila (\emph{stack overflow }), corrompiendo datos o provocando fallos.
  \item Llamadas infinitas o demasiadas llamadas pueden exceder límites del sistema operativo.
  \item Mitigaciones:
    \begin{itemize}
      \item Asegurar una condición base clara para limitar la profundidad.
      \item Reescribir en forma iterativa cuando convenga.
      \item Emplear recursión de cola (tail recursion) para reutilizar el mismo \emph{frame}.
      \item Incrementar el tamaño de pila si el entorno lo permite.
      \item Incluir chequeos que aborten antes de niveles críticos.
    \end{itemize}
\end{itemize}

\section{Diferencias en uso de memoria y registros: Iterativo vs. Recursivo}


\renewcommand{\arraystretch}{1.4}
\begin{center}
\begin{tabular}{|p{3.3cm}|p{5.7cm}|p{5.7cm}|}
\hline
\textbf{Aspecto técnico} & \textbf{Iterativo} & \textbf{Recursivo} \\
\hline
Registros usados & Utiliza registros temporales (\texttt{\$t0--\$t9}) que no necesitan respaldo. Simples de manejar, ideales para cálculos directos sin contaminación de contexto. & Emplea registros de argumento (\texttt{\$a0--\$a3}), de retorno (\texttt{\$ra}), resultado (\texttt{\$v0}) y de marco (\texttt{\$fp}, \texttt{\$sp}). Requiere respaldo y restauración meticulosa. \\
\hline
Uso de la pila & Pila opcional; sólo se usa si se requiere preservar datos entre iteraciones. Bajo riesgo de desbordamiento. & Cada llamada añade un marco de activación: dirección de retorno, parámetros, locales y registros. Alta presión sobre la pila, riesgo de desbordamiento de datos. \\
\hline
Eficiencia en memoria & Consumo constante y predecible. Apto para sistemas embebidos y estructuras iterativas simples. & Memoria dinámica y creciente con profundidad de llamadas. Puede volverse ineficiente sin optimización. \\
\hline
Control del flujo & Basado en saltos condicionales (\texttt{beq}, \texttt{bne}, etc.). Estructura clara y fácil de seguir. Ideal para trazado y análisis estático. & Uso de \texttt{jal} y \texttt{jr} para llamadas anidadas. Control del flujo implícito, menos transparente y más difícil de depurar. \\
\hline
Complejidad del código & Estructura lineal. Código explícito y fácil de mantener, aunque más largo en problemas jerárquicos. & Código conciso pero con mayor abstracción. Ideal para estructuras auto-similares como árboles o algoritmos de retroceso. \\
\hline
Velocidad de ejecución & Más veloz por evitar salto de contexto, respaldo y restauración. & Menor velocidad por sobrecarga de llamadas, manejo de pila y marcos de activación. \\
\hline
Errores comunes & Bucles infinitos si la condición de salida falla. Pérdida de control si se manipulan contadores incorrectamente. & Desbordamiento de datos si falta o no se alcanza la condición base. Pérdida de retorno si \texttt{\$ra} no se guarda correctamente. \\
\hline
Legibilidad & Muy clara para algoritmos secuenciales simples (e.g., suma acumulativa). & Mejor legibilidad en algoritmos que se describen naturalmente de forma recursiva (e.g., recorrido de árbol, combinatoria). \\
\hline
\end{tabular}
\end{center}

\newpage
\section{Diferencias entre ejemplos académicos y ejercicios en MIPS32}

     Mientras los ejercicios del libro solo requieren diseñar funciones específicas, implementarlas en un entorno de ejecución real como \texttt{MARS}, que es el que estamos usando, que exige construir un programa completo: incluyeendo declarar variables en la sección \texttt{.data}, codificar un punto de entrada \texttt{main}, gestionar manualmente los registros según convenciones (preservando \texttt{\$s0-\$s7}), y utilizar \textit{syscalls} para interactuar con el sistema , ya que aquí se debe controlar todo el flujo (desde la inicialización hasta la liberación de recursos) e integrar explícitamente operaciones que en el contexto teórico se asumen abstractas, culminando siempre con una llamada de salida (\textit{syscall} con \texttt{\$v0=10}) para evitar errores de ejecución.

\section{Elaborar un tutorial de la ejecución paso a paso en MARS}

\begin{itemize}
      \item \textbf{Paso 1:} Abrir un archivo .asm en MARS.
      
      \item \textbf{Paso 2:} Buscar en la barra de herramientas un botón con una llave y un destornillador formando una \textbf{X}. Sabrás que es el correcto cuando al pasar el cursor sobre él diga: \textit{``Ensamblar el archivo actual y eliminar los puntos de interrupción''}.
      
      \item \textbf{Paso 3:} Dos botones a la derecha de este, hay un botón con un círculo verde y un triángulo blanco apuntando a la izquierda, con un pequeño uno abajo. Sabrás que es el correcto cuando al pasar el cursor sobre él diga: \textit{``Ejecutar un paso a la vez''}.
      
      \item \textbf{Paso 4:} Si necesitas retroceder, a la izquierda del botón anterior hay un botón con un círculo azul y un triángulo blanco apuntando a la derecha, con un pequeño uno abajo. Sabrás que es el correcto cuando al pasar el cursor sobre él diga: \textit{``Regresar al último paso''}.
      
\end{itemize}


\section{Justificar la elección del enfoque (iterativo o recursivo) según eficiencia y claridad en
MIPS.}
\subsection*{Iterativo}

Optamos por el método iterativo ya que es más legible y sencillo de comprender; a diferencia de su versión recursiva, este método no necesita de llamadas recursivas ni el uso de pilas para no perder el valor de los datos. Ambos métodos son similares a la hora de pedir la entrada y las salidas, pero para realizar el cálculo del término n de la secuencia de Fibonacci, el método iterativo es más sencillo y fácil de intuir. En cambio, en el método recursivo es necesario el uso de pilas para guardar la dirección de memoria y otros datos necesarios para realizar las operaciones necesarias; y aunque ambos están en riesgo de desbordamiento aritmético, el método recursivo es propenso a tener desbordamiento en la pila por el simple hecho de utilizarla.

\section{Análisis y Discusión de los Resultados}

En el caso del Fibonacci iterativo, luego de tomar el n de la entrada verificamos que sea válido mediante un condicional que arrojará un mensaje de error en caso de que no sea válido, antes de entrar a los cálculos de la secuencia Fibonacci es necesario almacenar los primeros dos números de la serie y verificar que el n sea o no uno de esos números, luego entramos a la etiqueta donde se encuentra el ciclo que realiza las operaciones, finalmente imprimimos el valor de la serie y cerramos el programa.

Para el caso del Fibonacci recursivo el procedimiento es similar al iterativo, pero su diferencia radica en el cálculo del n-ésimo término. En este método se realizó un “jal” (jump and link) hacia la función recursiva; seguido se tuvo que cargar una pila con la dirección y los registros necesarios para almacenar los datos. Se realizan los cálculos aritméticos y de asignación necesarios, disminuyendo el n de la entrada hasta o que sea cero. De ahí saltamos a una etiqueta que descargará la pila almacenada hasta que llegó al caso base para que al final retornar a la función principal, imprimir la salida y cerrar el programa.

\end{document}
